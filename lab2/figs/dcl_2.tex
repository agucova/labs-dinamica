% basado en parte en https://tex.stackexchange.com/questions/453586/how-to-make-this-free-body-diagram-with-a-non-straight-force-line-using-tikzpict
\documentclass{standalone}

\usepackage{amsmath}
\usepackage{tikz, pgfplots}
\usetikzlibrary{arrows,
    calc,
    decorations,
    scopes,
}

\begin{document}
    \def\iangle{20} % Angle of the inclined plane

    \def\down{-90}
    \def\arcr{0.5cm} % Radius of the arc used to indicate angles

    \begin{tikzpicture}[
        >=latex',
        scale=1,
        force/.style={->,draw=blue,fill=blue},
        axis/.style={densely dashed,gray,font=\small},
        M/.style={rectangle,draw,fill=lightgray,minimum size=0.5cm,thin},
        m/.style={rectangle,draw=black,fill=lightgray,minimum size=0.3cm,thin},
        plane/.style={draw=black,fill=blue!10},
        string/.style={draw=red, thick},
        pulley/.style={thick},
        ]
        \pgfmathsetmacro{\Fnorme}{2}
        \pgfmathsetmacro{\Fangle}{30}
        \begin{scope}[rotate=\iangle]
            \node[M,transform shape] (M) {};
            \coordinate (xmin) at ($(M.south west)-({abs(1.1*\Fnorme*sin(-\Fangle))},0)$);
            \coordinate (xmax) at ($(M.south east)+({abs(1.1*\Fnorme*sin(-\Fangle))},0)$);
            \coordinate (ymax) at ($(M.north)+(0, {abs(1.1*\Fnorme*cos(-\Fangle))})$);
            \coordinate (ymin) at ($(M.south)-(0, 1cm)$);
            \draw[postaction={decorate, decoration={border, segment length=2pt, angle=-45},draw,red}] (xmin) -- (xmax);
            \coordinate (N) at ($(M.center)+(0,{\Fnorme*cos(-\Fangle)})$);
            \coordinate (fr) at ($(M.center)+({\Fnorme*sin(-\Fangle)}, 0)$);
            % Draw axes and help lines

            {[axis,->]
                \draw (ymin) -- (ymax) node[right] {$+y$};
                \draw (M) --(M-|xmax) node[right] {$+x$};    % mental note for me: change "right" to "above"
            }

            % Forces
            {[force,->]
                % Assuming that Mg = 1. The normal force will therefore be cos(alpha)
                \draw (M.center) -- (N) node [right] {$\vec N$};
            }
        \end{scope}
        % Draw gravity force. The code is put outside the rotated
        % scope for simplicity. No need to do any angle calculations.
        \draw[force,->] (M.center) -- ++(0,-1) node[below] {$\vec P$};
        \draw (M.center)+(-90:\arcr) arc [start angle=-90,end angle=\iangle-90,radius=\arcr] node [right, pos=.5] {};
    \end{tikzpicture}
\end{document}
